\section{Introduction}
\mylabel{sec:oper}

This chapter explains how to write image processing operations using the
VIPS image I/O (input-output) system. For background, you should probably
take a look at \pref{sec:appl}. This is supposed to be a tutorial, if you
need detailed information on any particular function, use the on-line UNIX
manual pages.

\subsection{Why use VIPS?}

If you use the VIPS image I/O system, you get a number of benefits:

\begin{description}

\item[Threading]
If your computer has more than one CPU, the VIPS I/O system will automatically
split your image processing operation into separate threads (provided you
use PIO, see below). You should get an approximately linear speed-up as
you add more CPUs.

\item[Pipelining]
Provided you use PIO (again, see below), VIPS can automatically join
operations together. A sequence of image processing operations will all
execute together, with image data flowing through the processing pipeline
in small pieces. This makes it possible to perform complex processing on
very large images with no need to worry about storage management.

\item[Composition]
Because VIPS can efficiently compose image processing operations, you can
implement your new operation in small, reusable, easy-to-understand
pieces. VIPS already has a lot of these: many new operations can be
implemented by simply composing existing operations.

\item[Large files]
Provided you use PIO and as long as the underlying OS supports large files
(that is, files larger than 2GB), VIPS operations can work on files larger
than can be addressed with 32 bits on a plain 32-bit machine. VIPS operations
only see 32 bit addresses; the VIPS I/O system transparently maps these to
64 bit operations for I/O. Large file support is included on most unixes after
about 1998.

\item[Abstraction]
VIPS operations see only arrays of numbers in native format. Details of
representation (big/little endian, VIPS/TIFF/JPEG file format, etc.) are 
hidden from you.

\item[Interfaces]
Once you have your image processing operation implemented, it automatically
appears in all of the VIPS interfaces. VIPS comes with a GUI (\nip{}), a
UNIX command-line interface (\vips{}) and a C++ and Python API.

\item[Portability]
VIPS operations can be compiled on most unixes, Mac OS X and Windows NT, 2000
and XP without modification. Mostly.

\end{description}

\subsection{I/O styles}

The I/O system supports three styles of input-output. 

\begin{description}

\item[Whole-image I/O (WIO)]
This style is a largely a left-over from VIPS 6.x. WIO image-processing
operations have all of the input image given to them in a large memory
array. They can read any of the input pels at will with simple pointer
arithmetic.

\item[Partial-image I/O (PIO)]
In this style operations only have a small part of the input image available
to them at any time. When PIO operations are joined together into a pipeline,
images flow through them in small pieces, with all the operations in a
pipeline executing at the same time. 

\item[In-place]
The third style allows pels to be read and written anywhere in
the image at any time, and is used by the VIPS in-place operations, such
as \verb+im_fastline()+. You should only use it for operations which would
just be impossibly inefficient to write with either of the other two styles.

\end{description}

WIO operations are easy to program, but slow and inflexible when images
become large. PIO operations are harder to program, but scale well as images
become larger, and are automatically parallelized by the VIPS I/O system.

If you can face it, and if your algorithm can be expressed in this way, you
should write your operations using PIO.  Whichever you choose, applications
which call your operation will see no difference, except in execution speed.

If your image processing operation performs no coordinate transformations,
that is, if your output image is the same size as your input image or images,
and if each output pixel depends only upon the pixel at the corresponding
position in the input images, then you can use the \verb+im_wrapone()+
and \verb+im_wrapmany()+ operations. These take a simple buffer-processing
operation supplied by you and wrap it up as a full-blown PIO operation. 
See~\pref{sec:wrapone}.

\subsection{What's new in this version}

The VIPS API is mostly unaltered since 7.3, so there are not many major
changes. I've just reworked the text, reformatted, fixed a few typos,
and changed the dates.

VIPS has acquired some crud over the years. We are planning to clean all
this stuff up at some stage (and break backwards-compatibility). Maybe for
VIPS 8 :-(
